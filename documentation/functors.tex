%%%%%%%%%%%%%%%%%%%%%%%%%%%%%%%%%%%%%%%%%%%%%%%%%%%%%%%%%%%%%%%%%%%%%%%%%%%%%%%%%%%%%%%%%%%%%%%%%%%
%
% Primordial Machine's Math Indices Library
% Copyright (C) 2017-2019 Michael Heilmann
%
% This software is provided 'as-is', without any express or implied warranty.
% In no event will the authors be held liable for any damages arising from the
% use of this software.
%
% Permission is granted to anyone to use this software for any purpose,
% including commercial applications, and to alter it and redistribute it
% freely, subject to the following restrictions:
%
% 1. The origin of this software must not be misrepresented;
%    you must not claim that you wrote the original software.
%    If you use this software in a product, an acknowledgment
%    in the product documentation would be appreciated but is not required.
%
% 2. Altered source versions must be plainly marked as such,
%    and must not be misrepresented as being the original software.
%
% 3. This notice may not be removed or altered from any source distribution.
%
%%%%%%%%%%%%%%%%%%%%%%%%%%%%%%%%%%%%%%%%%%%%%%%%%%%%%%%%%%%%%%%%%%%%%%%%%%%%%%%%%%%%%%%%%%%%%%%%%%%

%%%%%%%%%%%%%%%%%%%%%%%%%%%%%%%%%%%%%%%%%%%%%%%%%%%%%%%%%%%%%%%%%%%%%%%%%%%%%%%%%%%%%%%%%%%%%%%%%%%
\section{\textit{concept Functor}}
A functor is a struct type providing a constant  \texttt{operator()}.
That operator shall be qualified as \texttt{noexcept}     if possible
and shall be qualified as \texttt{constexpr} if possible.  The return
type of \texttt{operator()} shall be of type \texttt{result}    (or a
cv-qualified variant of that).\newline\noindent{}The functor    shall
provide the member type definition \texttt{result\_type} denoting the
type \texttt{result}.

%%%%%%%%%%%%%%%%%%%%%%%%%%%%%%%%%%%%%%%%%%%%%%%%%%%%%%%%%%%%%%%%%%%%%%%%%%%%%%%%%%%%%%%%%%%%%%%%%%%
\section{\textit{concept BinaryFunctor}}
A \textit{BinaryFunctor} is a \textit{Functor}.
Its \texttt{operator()} has two parameters \texttt{left\_operand} of type
\texttt{left\_operand\_type} (or a cv-qualified variant of that)      and 
\texttt{right\_operand} of type \texttt{right\_operand\_type} (or       a
cv-qualified variant of that).\newline

\noindent{}The functor shall provide the member type              definitions
\texttt{left\_operand\_type}  denoting the type \texttt{left\_operand}    and
\texttt{right\_operand\_type} denoting the type      \texttt{right\_operand}.
\newline

\section{\textit{concept BinaryFunctorBase}}
A  \textit{BinaryFunctorBase} is a template struct type of   name \textit{name}
with three template parameters \texttt{LEFT\_OPERAND}, \texttt{RIGHT\_OPERAND},
and \texttt{ENABLED}. The default value of \texttt{ENALBED} is   \texttt{void}.
Specializations of this template may use \texttt{ENABLED} to    perform SFINAE.
A possible implementation is
\texttt{\newline
\noindent{}template\textlangle typename LEFT\_OPERAND, typename RIGHT\_OPERAND,
typename ENABLED = void\textrangle\newline\noindent{}struct \textit{name};    }
\newline

\noindent{}Its partial specializations are \textit{BinaryFunctors}.
%%%%%%%%%%%%%%%%%%%%%%%%%%%%%%%%%%%%%%%%%%%%%%%%%%%%%%%%%%%%%%%%%%%%%%%%%%%%%%%%%%%%%%%%%%%%%%%%%%%
\section{\textit{concept UnaryFunctor}}
An \textit{UnaryFunctor} is \textit{Functor}.
Its \texttt{operator()} has one parameter \texttt{operand}  of type
\texttt{operand\_type} (or a cv-qualified variant of that).\newline

\noindent{}The functor shall provide the member type definition
\texttt{operand\_type} denoting the type      \texttt{operand}.

\section{\textit{concept UnaryFunctorBase}}
An \textit{UnaryFunctorBase} is a template struct type of     name \textit{name}
with two  template parameters \texttt{OPERAND} and \texttt{ENABLED}. The default
value of \texttt{ENALBED} is \texttt{void}. Specializations of this template may
use \texttt{ENABLED} to perform SFINAE.
A possible implementation is
\texttt{\newline
\noindent{}template\textlangle typename OPERAND,
typename ENABLED = void\textrangle\newline\noindent{}struct \textit{name};    }
\newline

\noindent{}Its partial specializations are \textit{UnaryFunctors}.
