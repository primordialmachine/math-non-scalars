%%%%%%%%%%%%%%%%%%%%%%%%%%%%%%%%%%%%%%%%%%%%%%%%%%%%%%%%%%%%%%%%%%%%%%%%%%%%%%%%%%%%%%%%%%%%%%%%%%%
%
% Primordial Machine's Math Non Scalars Library
% Copyright (C) 2017-2019 Michael Heilmann
%
% This software is provided 'as-is', without any express or implied warranty.
% In no event will the authors be held liable for any damages arising from the
% use of this software.
%
% Permission is granted to anyone to use this software for any purpose,
% including commercial applications, and to alter it and redistribute it
% freely, subject to the following restrictions:
%
% 1. The origin of this software must not be misrepresented;
%    you must not claim that you wrote the original software.
%    If you use this software in a product, an acknowledgment
%    in the product documentation would be appreciated but is not required.
%
% 2. Altered source versions must be plainly marked as such,
%    and must not be misrepresented as being the original software.
%
% 3. This notice may not be removed or altered from any source distribution.
%
%%%%%%%%%%%%%%%%%%%%%%%%%%%%%%%%%%%%%%%%%%%%%%%%%%%%%%%%%%%%%%%%%%%%%%%%%%%%%%%%%%%%%%%%%%%%%%%%%%%

\documentclass[oneside]{book}

\input{common}
\SetOrganization{Primordial Machine}
\SetLibraryName{Math Non Scalars}
\SetLibraryVersion{1.2}
\SetLibraryRepository{https://github.com/primordialmachine/math-non-scalars}
\SetAuthor{Michael Heilmann}
\SetEmail{michaelheilmann@primordialmachine.com}


\SetLibraryIncludeFileName{include.hpp}
\SetLibraryIncludesDirectoryPath{primordialmachine/math-non-scalars/\newline\$(PlatformTarget.toLower())/\$(Configuration.toLower())/includes}

\SetLibraryIncludeDirectiveFilePath{primordialmachine/math/non\_scalars/include.hpp}

\SetLibraryStaticLibrariesDirectoryPath{primordialmachine/math/non-scalars/\newline\$(PlatformTarget.toLower())/\$(Configuration.toLower())/libraries}
\SetLibraryStaticLibraryFileName{math-non-scalars.lib}

\SetDocumentType{User Manual}

\usepackage[listings,theorems]{tcolorbox}
\tcbset{before={\par\medskip\pagebreak[0]\noindent},after={\par\medskip}}%
\newcounter{example}[section]

\begin{document}

\frontmatter

\begin{titlepage}
\maketitle
\end{titlepage}

\tableofcontents
\addtocontents{toc}{\protect\thispagestyle{empty}}
\pagenumbering{gobble}

\mainmatter

\chapter{Synopsis}
C++ 17 library providing concepts and traits related to mathematical non-scalar types.
The library is made available publicly on
\href{\GetLibraryRepository}{Github}
under the
\href{\GetLibraryRepository/blob/master/LICENSE}{MIT License}.

\chapter{Limitations and Restrictions}
The library officially only supports Visual Studio 2017 and Windows 10.

\chapter{Introductory example}
\textit{\color{orange}This library does not provide any examples yet.}
%Examples are located in the \href{\GetLibraryRepository/blob/master/examples}{examples} directory.

\input{building_visual_studio_2017}

\chapter{Library Interface Documentation}

\section{\texttt{namespace primordialmachine}}
The namespace this library is adding its declarations/definitions to.
The added namespace elements are documented below.

\section{\textit{Math.NonScalar} (type concept)}
A \textit{Math.NonScalar} type is a fixed size container for   \textit{Math.Scalar}
type elements which can be accessed for reading an writing using different indexing
schemes. Matrices, points, and vectors are examples of      \textit{Math.NonScalar}
types. 

\subsection{Element type}
The \textit{Math.Scalar} type $E$ of the elements of a \textit{Math.NonScalar} type
$T$ can be obtained by the member type definition\newline
\texttt{element\_type\textlangle $T$\textrangle::type}\newline
or the type definition\newline
\texttt{element\_type\_t\textlangle$T$\textrangle}\newline

\subsection{Number of elements}
The number of elements of a \textit{Math.NonScalar} type $T$ can be obtained by
the constant expression \textit{size\_t} member constant\newline
\texttt{number\_of\_elements\textlangle $T$\textrangle::value}\newline
or by the constant expression \textit{size\_t} constant\newline
\texttt{number\_of\_elements\_v\textlangle $T$\textrangle}\newline

\subsection{Degenerate and non-degenerate}
A \textit{Math.NonScalar} type $T$ called \textit{degenerate} if and only if it has zero elements.
Otherwise it is called \textit{non-degenerate}.

If a \textit{Math.NonScalar} type $T$ is degenerate can be obtained by
the constant expression \textit{bool} member constant\newline
\texttt{is\_degenerate\textlangle $T$\textrangle::value}\newline
or by the constant expression \textit{bool} constant\newline
\texttt{is\_degenerate\_v\textlangle $T$\textrangle}\newline

If a \textit{Math.NonScalar} type $T$ is non-degenerate can be obtained by
the constant expression \textit{bool} member constant\newline
\texttt{is\_non\_degenerate\textlangle $T$\textrangle::value}\newline
or by the constant expression \textit{bool} constant\newline
\texttt{is\_non\_degenerate\_v\textlangle $T$\textrangle}\newline

\subsection{Dimensionality $=0$}
A \textit{Math.NonScalar} of dimensionality $=0$ is a type $T$
which provides support for the following   operations:\newline

%BEGIN OPERATIONS%
\texttt{$T$(const $T$\& other)}\textit{(copy constructor)}\newline
\texttt{$T$\& operator=(const $T$\& other)}\textit{(copy assignment operator)}\newline
%END OPERATIONS%

\subsection{Dimensionality $=1$}
A \textit{Math.NonScalar} of dimensionality $=1$ provides the     same
operations as a \textit{Math.NonScalar} type of   dimensionality $=0$.
In addition, it provides support for the following operations:\newline

%BEGIN OPERATIONS%
\texttt{const element\_type\& operator()(size\_t i) const}\textit{(constant unary indexing operator})\newline
\texttt{element\_type\& operator()(size\_t i)} \textit{(unary indexing operator)}\newline
%END OPERATIONS%

\subsection{Dimensionality $>1$}
A \textit{Math.NonScalar} of dimensionality $>1$ provides the     same
operations as a \textit{Math.NonScalar} type of   dimensionality $=1$.
In addition, it provides support for the following operations:\newline

%BEGIN OPERATIONS%
\texttt{const element\_type\& operator()(\textit{IndexList}) const} \textit{(constant $n > 1$-ary indexing operator)}\newline
\texttt{element\_type\& operator()(\textit{IndexList})} \textit{($n > 1$-ary indexing operator)}\newline
%END OPERATIONS%

where \texttt{\textit{IndexList}} are $n$ \texttt{size\_t} values
denoting indices and where $n$ is the dimensionality of      $T$.\newline

Every \textit{Math.NonScalar} of dimensionality $n > 1$     must define a
bijection between the indices $i$ passed to the unary \texttt{operator()}
and the indices $i_0,\ldots,i_{n-1}$ passed to the            $n > 1$-ary
\texttt{operator()}.\newline

\begin{tcolorbox}[title=Example]
Row-major $N \times M$ matrices where $N$ is the   number of
rows and $M$ is the number of columns can fulfil         the
\textit{Math.NonScalar} type concept of  dimensionality $2$:
First, provide th $2$ index operators such that the first
index denotes the row index and the second index      the
column index.
Second, provide the $1$ index   operators and define the
bijection between an index $i$ passed to the $1$   index
operator and the first index $j$   and the second index:
A possible bijection can be defined in terms of a    row
major mapping as $k$ as $i / N + i\mod M= j \cdot M +k$.
\end{tcolorbox}

\subsection{Number of elements}
\textcolor{orange}{To be done.}

%%%%%%%%%%%%%%%%%%%%%%%%%%%%%%%%%%%%%%%%%%%%%%%%%%%%%%%%%%%%%%%%%%%%%%%%%%%%%%%%%%%%%%%%%%%%%%%%%%%
\section{\texttt{index\_1} (struct)}
A \texttt{struct} representing a one dimensional index.

\subsection{Constructors}
\subsubsection{\texttt{index\_1(size\_t i)}}
\texttt{constexpr index\_1(size\_t i)}\newline
Contruct this \texttt{index\_1} object with the specified index.

\subsection{Methods}
\subsubsection{\texttt{i()}}
\texttt{constexpr const size\_t\& i() const}\newline
\texttt{constexpr size\_t\& i()}\newline
Method returning the index \texttt{i}.

%%%%%%%%%%%%%%%%%%%%%%%%%%%%%%%%%%%%%%%%%%%%%%%%%%%%%%%%%%%%%%%%%%%%%%%%%%%%%%%%%%%%%%%%%%%%%%%%%%%
\section{\texttt{index\_2} (struct)}
A \texttt{struct} representing a two dimensional index.

\subsection{Constructors}
\subsubsection{\texttt{index\_2(size\_t i, size\_t j)}}
\texttt{index\_1(size\_t i, size\_t j)}\newline
Contruct this \texttt{index\_2} object with the specified indices.

\subsection{Methods}
\subsubsection{\texttt{i()}}
\texttt{constexpr const size\_t\& i() const}\newline
\texttt{constexpr size\_t\& i()}\newline
Method returning the index \texttt{i}.

\subsubsection{\texttt{j()}}
\texttt{constexpr const size\_t\& j() const}\newline
\texttt{constexpr size\_t\& j()}\newline
Method returning the index \texttt{i}.

\end{document}
